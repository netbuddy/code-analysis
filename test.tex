\documentclass[12pt]{article}
\usepackage{ctex} % 支持中文
\usepackage[a4paper, margin=1in]{geometry} % 页面布局

\begin{document}

\title{2023年小学六年级数学试卷}
\author{}
\date{}
\maketitle

\section*{一、计算,(35分)}

\subsection*{1. 口算(8分)}
\begin{enumerate}
    \item $0.125 \times 5 \times 0.8$
    \item $6.4 + 0.8$
    \item $50 \times 40$
    \item $4.6 + 4$
    \item $3$
    \item $43$
    \item $11$
    \item $A$
    \item $77$
    \item $43$
\end{enumerate}

\subsection*{2. 计算下面各题,(18分)}
\begin{enumerate}
    \item[(2)] $519.3 - (19.3 - 6.7)$
    \item[(1)] $26 \times 35 + 5.4 + 0.27$
    \item[(4)]
    \item[(3)] $24$
    \item 3
    \item 6
    \item 5
    \item[(6)]
\end{enumerate}

\subsection*{3. 解方程,(9分)}
\begin{enumerate}
    \item[(2)] $2x - 0.36 \times 5 - 0.4$
    \item[(3)] $2 \times (1.7 - x) \times 0.4$
    \item 8
\end{enumerate}

\section*{二、填空(20分)}
\subsection*{4. 选择合适的数填入括号里。(每个数只能选一次)}
\begin{itemize}
    \item 8590
    \item 1
    \item 5
    \item 95.5
    \item 6
    \item 5
\end{itemize}
(1)小明期末考试语文、数学、英语三科成绩的平均分是90分。三科中数学成绩最好,数学考了( 分 )

(2)把一根长5分米的小棒平均分成6份,每份长( )分米,

5.2升750毫升=( )升

3.2时=3时( )分

635( )%

\subsection*{5. 根据题目所给信息填空}
一个三角形的两个内角之和等于78°,这个三角形的第三个内角是( ),根据三角形的分类,它是一个( )三角形。

如果一个数的循环小数表示为0.428571428571428571,这个循环小数用简便形式写是( ),小数点后第2023位的数字是( )。

地上有×吨水泥,每天用1.9吨,用了x天后,用了( )吨。照这样计算,这些水泥还可以用( )天。

如果购买的黄瓜是45kg,多范子有( )kg。

男生人数是女生人数的( )倍。

\subsection*{6. 解决问题}
(1)如果这个圆柱体的底面半径是2分米,高是5分米,那么这个圆柱体的体积是( )立方分米。

(2)觉得空载一辆线感的长度约是35分米,这个地图的比例尺是( )。

(3)天空会开车,从起点站九峰山站出发,以60千米/时的平均速度行驶,经过x分钟后可到达路点站就象山站。

\section*{三、正确的序号填入括号中。(8分)}
\begin{enumerate}
    \item[14] 新米停号
    \item[15] 第六次全国人口普查统计显示全国人口141178亿,数字下的计数单位是( )。
    \item[16] 用7和2组成的两位数一定是( )。
    \item[17] 从上面看下面的物体,形状与其他三项不相同的是( )。
    \item[18] 好妈将1000元存入银行,存期3年,年利率为2.75\%,到期后可得利息( )元。
    \item[19] 二只不透明的布袋中装了一些小球,这些小球材质、大小完全相同,只是颜色不同,小玲每次摸出一个球,放回搅匀后再摸,摸了6次,每次摸到的是白球,下面说法合理的是( )。
    \item[20] 将一张圆形纸对折再对折,然后在中间剪一个"S"形(如图),再将它展开,展开后的图形是( )。
    \item[21] 用汽车搬运一批货物,已经运了5次(每次都装满),运走的货物比总量的1/3多一些,比总量的1/2少一些,运完这批货物最少要( )次。
\end{enumerate}

\section*{四、操作。(5分)}
\subsection*{22. 根据题目所给信息作答}
(1)点A在点C的( )方向上(X)。
(2)如果B点的位置用数对表示是(5,2),那么D点的位置用数对表示是(,)。
(3)将这个图形绕点A顺时针旋转90°,画出旋转后的图形。

\section*{五、解决问题(32分)}
\subsection*{23. 根据题目所给信息解决问题}
5月30日神舟十六号载人飞船成功发射升空,据悉,新型舱内航天服重约20千克,舱外航天服重约120千克,舱内航天服比舱外航天服轻百分之多少?

\subsection*{24. 解决问题}
六(1)班57名同学一起拍毕业集体照,拍摄费用100元(含5张照片),每加印一张照片2.5元,如果每名同学都需要一张照片,共付多少钱?

\subsection*{25. 解决问题}
东东的身高是1.5米,他的影子长2.4米,如果同一时间,同一地点测得一棵树的影子长8米,这棵树有多高?

\subsection*{26. 解决问题}
一项工程,甲队独做12天完成,乙队独做要15天,如果乙队先单独做工程的1/10,剩下的再由甲、乙合作,那么做完这个工程共用多少天?

\subsection*{27. 解决问题}
下面是武汉到南京的长江航线示意图。
\begin{itemize}
    \item 武汉一九江 270千米
    \item 武汉一芜湖 637千米
    \item 武汉一南京 733千米
\end{itemize}
(1)“733-637”计算的是哪两个城市之间的航程?
(2)两艘轮船同时从武汉、九江两地开出,相向而行,经过3.6小时相遇。甲船每小时行38千米,乙船每小时行多少千米?

\subsection*{28. 解决问题}
一个长方体无盖玻璃鱼缸,长是7分米,宽是2分米。鱼缸中放着一块高3分米,体积为6立方分米的假山石。如果水管以每分钟9立方分米的流量向鱼缸中注水,至少需要多少分钟才能将假山石刚好淹没?

\end{document}
